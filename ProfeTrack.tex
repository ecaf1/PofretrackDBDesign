\documentclass{article}
\usepackage[utf8]{inputenc}
\usepackage{geometry}
\geometry{a4paper, margin=1in}
\usepackage{hyperref}
\usepackage{enumitem}

\title{ProfeTrack: Especificações do Projeto}
\author{}
\date{}

\begin{document}

\maketitle

\section{Introdução}
O ProfeTrack é uma aplicação web destinada a facilitar o monitoramento e gerenciamento do planejamento de aulas por coordenadores escolares em instituições de ensino fundamental e médio. A aplicação visa proporcionar uma interface intuitiva para que os coordenadores possam visualizar, comentar e monitorar os planejamentos de aula dos professores, além de permitir que os professores gerenciem seus próprios planejamentos de forma eficiente. Dentro do gerenciamento de planejamento feito por um coordenador existe a possibilidade de realizar comentários e notificar o respectivo professor sobre planejamentos atrasados ou com alguma pendência.

\section{Características Essenciais para Implementação no Banco de Dados}

\subsection{Características de Usuário}
\begin{itemize}
    \item É crucial visar a simplicidade na criação de um novo perfil de usuário, destacando-se dois tipos distintos de usuários: professores e coordenadores, sendo que um coordenador também pode desempenhar o papel de professor.
\end{itemize}

\subsection{Cadastro de Disciplinas}
\begin{itemize}
    \item Ao considerar o cadastro de disciplinas, algumas características importantes incluem:
    \begin{itemize}
        \item Nome da disciplina;
        \item Professores vinculados;
        \item Ano letivo;
        \item Série;
        \item Turno;
        \item Nível (1, 2, 3).
    \end{itemize}
\end{itemize}

\subsection{Criação de Planejamento de Aula}

\begin{itemize}
    \item Ao considerar o planejamento de aula, algumas informações importantes incluem:
    \begin{itemize}
        \item Data;
        \item Nº do Módulo;
        \item Nº da Aula;
        \item Conteúdo;
        \item Meta de Aula;
        \item Metodologia Aplicada;
        \item Atividades Resolvidas em Sala;
        \item Atividades para Casa;
        \item Vinculação com a Disciplina.
    \end{itemize}
\end{itemize}

\subsection{Criação de Comentários}
\begin{itemize}
    \item É necessário apenas que o comentário seja feito e direcionado aos professores responsáveis pela disciplina.
\end{itemize}
\end{document}
