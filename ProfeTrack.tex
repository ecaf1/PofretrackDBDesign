\documentclass{article}
\usepackage[utf8]{inputenc}
\usepackage{geometry}
\geometry{a4paper, margin=1in}
\usepackage{hyperref}
\usepackage{enumitem}

\title{ProfeTrack: Especificações do Projeto}
\author{}
\date{}

\begin{document}

\maketitle

\section{Introdução}
O ProfeTrack é uma aplicação web destinada a facilitar o monitoramento e gerenciamento do planejamento de aulas por coordenadores escolares em instituições de ensino fundamental e médio. A aplicação visa proporcionar uma interface intuitiva para que os coordenadores possam visualizar, comentar e monitorar os planejamentos de aula dos professores, além de permitir que os professores gerenciem seus próprios planejamentos de forma eficiente. Dentro do gerenciamento de planejamento feito por um coordenador existe a possibilidade de realizar comentários e notificar o respectivo professor sobre planejamentos atrasados ou com alguma pendência.

\section{Características Essenciais para Implementação no Banco de Dados}

\subsection{Características de Usuário}
\begin{itemize}
    \item É crucial visar a simplicidade na criação de um novo perfil de usuário, destacando-se dois tipos distintos de usuários: professores e coordenadores, sendo que um coordenador também pode desempenhar o papel de professor.
\end{itemize}

\subsection{Cadastro de Disciplinas}
\begin{itemize}
    \item Ao considerar o cadastro de disciplinas, algumas características importantes incluem:
    \begin{itemize}
        \item Nome da disciplina;
        \item Professores vinculados;
        \item Ano letivo;
        \item Série;
        \item Turno;
        \item Nível (1, 2, 3).
    \end{itemize}
\end{itemize}

\subsection{Criação de Planejamento de Aula}

\begin{itemize}
    \item Ao considerar o planejamento de aula, algumas informações importantes incluem:
    \begin{itemize}
        \item Data;
        \item Nº do Módulo;
        \item Nº da Aula;
        \item Conteúdo;
        \item Meta de Aula;
        \item Metodologia Aplicada;
        \item Atividades Resolvidas em Sala;
        \item Atividades para Casa;
        \item Vinculação com a Disciplina.
    \end{itemize}
\end{itemize}

\subsection{Criação de Comentários}
\begin{itemize}
    \item É necessário apenas que o comentário seja feito e direcionado aos professores responsáveis pela disciplina.
\end{itemize}
\subsection{Normalização de Tabelas}
A estruturação das tabelas do banco de dados segue rigorosamente os princípios de normalização, com o objetivo de otimizar a integridade e a eficiência dos dados. A seguir, detalhamos como cada tabela atende aos critérios das três formas normais (1NF, 2NF e 3NF).

\subsubsection{1ª Forma Normal (1NF)}
A 1NF requer que:
\begin{itemize}
    \item Todos os valores de colunas sejam atômicos, ou seja, indivisíveis.
    \item A tabela deve possuir uma chave primária única.
\end{itemize}

\textbf{Justificativa:}
\begin{itemize}
    \item \textit{Tabela usuarios}: Cumpre a 1NF com valores atômicos e uma chave primária (\textit{id}), garantindo registros únicos.
    \item \textit{Tabela usuario\_tipo}: Segue a 1NF com valores atômicos. Usa \textit{usuario\_id} e \textit{tipo} para identificação única, assegurada por uma chave estrangeira referenciando \textit{usuarios}.
    \item \textit{Tabela disciplinas}: Em conformidade com a 1NF por ter valores atômicos e uma chave primária (\textit{id}).
    \item \textit{Tabela planejamentos}: Adere à 1NF, possuindo valores atômicos e identificação única pela chave primária (\textit{id}).
    \item \textit{Tabela comentario}: Observa a 1NF com valores atômicos e relação com \textit{planejamentos} através de chave estrangeira.
    \item \textit{Tabela usuario\_disciplina}: Atende a 1NF com valores atômicos e uma chave primária (\textit{id}).
\end{itemize}

\subsubsection{2ª Forma Normal (2NF)}
A 2NF é aplicada a tabelas já em 1NF e requer que:
\begin{itemize}
    \item A tabela esteja livre de dependências parciais, ou seja, todos os atributos não-chave devem depender totalmente da chave primária.
\end{itemize}

\textbf{Justificativa:}
\begin{itemize}
    \item \textit{Tabelas usuarios, disciplinas, e usuario\_disciplina}: Estão na 2NF, com atributos não-chave dependendo unicamente das chaves primárias.
    \item \textit{Tabela planejamentos e comentario}: Também cumprem a 2NF, com todas as colunas não-chave dependendo integralmente da chave primária para identificação.
    \item \textit{Tabela usuario\_tipo}: Sugerida a conformidade com a 2NF, apesar da ausência de uma chave primária composta explicitamente definida.
\end{itemize}

\subsubsection{3ª Forma Normal (3NF)}
A 3NF é alcançada quando a tabela está em 2NF e:
\begin{itemize}
    \item Não possui dependências transitivas; ou seja, atributos não-chave dependem apenas da chave primária, e não de outros atributos não-chave.
\end{itemize}

\textbf{Justificativa:}
\begin{itemize}
    \item \textit{Todas as tabelas}: Não apresentam dependências transitivas, com cada atributo não-chave dependendo diretamente da chave primária da sua tabela.
\end{itemize}


    
\end{document}
